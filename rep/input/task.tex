\begin{center}
   \textbf{\titlePageWorkType~№\titlePageWorkNumber}
\end{center}

\subparagraph{Тема:} <<\titlePageTopic>>

\subparagraph{Цель работы:} Научиться программировать ввод и вывод в С++, используя объекты потоковых классов сдандартной библиотеки С++.

\begin{center}
    \textbf{Вариант 5}:
 \end{center}

\begin{center}
   \textbf{Ход работы}:
\end{center}

\paragraph{1. Подстановка задачи}\hspace{0pt}

    Напишите программу, которая печатает слова из файла, расположенные в порядке убывания частоты их появления. Перед каждым словом напечатайте число его появлений.

\paragraph{2. Программа решения задания}\hspace{0pt}

    \VerbatimInput[
        breaklines=true,
        breakanywhere=true,
        fontfamily=courier,
        frame=single,
        framesep=4mm,
        labelposition=all,
        numbers=left,
        label=main.cpp,
        fontsize=\small,
    ]
    {../../src/CodeBlocks_4-semestr-OOTPiSP_lab-2/main.cpp}

    \VerbatimInput[
        breaklines=true,
        breakanywhere=true,
        fontfamily=courier,
        frame=single,
        framesep=4mm,
        labelposition=all,
        numbers=left,
        label=task\_5.hpp,
        fontsize=\small,
    ]
    {../../src/CodeBlocks_4-semestr-OOTPiSP_lab-2/task_5/task_5.hpp}

    \VerbatimInput[
        breaklines=true,
        breakanywhere=true,
        fontfamily=courier,
        frame=single,
        framesep=4mm,
        labelposition=all,
        numbers=left,
        label=task\_5.cpp,
        fontsize=\small,
    ]
    {../../src/CodeBlocks_4-semestr-OOTPiSP_lab-2/task_5/task_5.cpp}

\newpage

\paragraph{3. Результат работы программы}\hspace{0pt}

    \VerbatimInput[
        breaklines=true,
        breakanywhere=true,
        fontfamily=courier,
        frame=single,
        framesep=4mm,
        labelposition=all,
        numbers=left,
        label=text.txt,
        fontsize=\small,
    ]
    {../../src/CodeBlocks_4-semestr-OOTPiSP_lab-2/text.txt}

    \begin{MyVerbatimCode}[
        label=Console out,
        fontsize=\small,
    ]
4	"banana"
3	"apple"
2	"pear"
2	"avocado"
1	"plum"
1	"mango"
1	"pomelo"
1	"watermelon"
1	"kiwi"
1	"date"
\end{MyVerbatimCode}

\subparagraph{Вывод:} Изучили метод чтения файла в стиле С++, используя библиотеку fstream.
